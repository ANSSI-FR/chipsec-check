\hypertarget{usb-key-building-for-chipsec-and-secureboot-checks-v0.1-042019}{%
\section{USB KEY BUILDING FOR CHIPSEC AND SECUREBOOT CHECKS v0.1
(04/2019)}\label{usb-key-building-for-chipsec-and-secureboot-checks-v0.1-042019}}

\begin{verbatim}
A Help to build your own ChipSec and SecureBoot USB keys
\end{verbatim}

\hypertarget{final-usb-keys}{%
\subsection{Final USB keys}\label{final-usb-keys}}

\begin{itemize}
\tightlist
\item
  \textbf{USB KEY 1} : live Debian distribution to launch
  \textbf{ChipSec} from the computer to analyze
\item
  \textbf{USB KEY 2} : contains \textbf{SecureBoot} keys to import, tool
  to import your own trust keys and to check importation
\end{itemize}

\hypertarget{linux-tools-to-install-before-generating-the-usb-keys}{%
\subsection{Linux Tools to install before generating the USB
keys}\label{linux-tools-to-install-before-generating-the-usb-keys}}

\begin{verbatim}
    sudo apt-get install debootstrap
    sudo apt-get install sbsigntool
    sudo apt-get install efitools
\end{verbatim}

\hypertarget{tool-to-build-the-usb-keys-create-keys.sh}{%
\subsection{Tool to build the usb keys :
create-keys.sh}\label{tool-to-build-the-usb-keys-create-keys.sh}}

\hypertarget{before-launching}{%
\subsubsection{Before launching}\label{before-launching}}

\begin{itemize}
\tightlist
\item
  define ``mount\_point'' variable (path to mount point) into
  ``create-shell.sh'' file
\item
  ensure that path ``mount\_point'' is empty and available
\end{itemize}

\hypertarget{build-usb-key-1}{%
\subsubsection{Build USB KEY 1}\label{build-usb-key-1}}

Plug a new usb key (attached on /dev/sdc in this case).

\begin{verbatim}
./create-keys.sh /dev/sdc -
\end{verbatim}

Unplug the usb key.

\hypertarget{build-usb-key-2}{%
\subsubsection{Build USB KEY 2}\label{build-usb-key-2}}

Plug a new usb key (attached on /dev/sdc in this case).

\begin{verbatim}
./create-keys.sh - /dev/sdc
\end{verbatim}

Unplug the usb key.

\hypertarget{boot-on-keys}{%
\subsection{Boot on keys}\label{boot-on-keys}}

\begin{itemize}
\item
  Plug one of keys, start the computer.
\item
  For USB KEY 1 :

  \begin{itemize}
  \tightlist
  \item
    1/ boot on usb key, start linux live
  \item
    2/ from root terminal, launch ChipSec with ``chipsec\_main.py''.
  \end{itemize}
\item
  For USB KEY 2 :

  \begin{itemize}
  \tightlist
  \item
    1/ boot on usb key and launch EFI binaries from EFI shell (Shell.efi
    is automaticaly started).
  \item
    OR interrupt the normal boot to select a shell EFI from Boot
    Configuration and launch EFI binaries from EFI shell.
  \item
    2/ launching of binaries from EFI shell :

    \begin{itemize}
    \tightlist
    \item
      Before to launch the binaries, it is imperative to identify the
      usb key letter storing the binaries with commmands ``fs0'' or
      ``fs1'' or fsX \ldots{} then ``dir''
    \item
      After disabling of SecureBoot and enabling of Setup Mode (with
      BIOS options) : launch ``KeyTool.efi'' to import trust keys.
    \item
      After re-enabling of SecureBoot and disabling of Setupe Mode (with
      BIOS options) : launch ``HelloWorld.efi'' (signed with imported
      Trust Keys) to check the good importation of trust keys.
    \end{itemize}
  \end{itemize}
\end{itemize}
