\hypertarget{chipsec-guide-v0.1-042019}{%
\section{CHIPSEC GUIDE v0.1 (04/2019)}\label{chipsec-guide-v0.1-042019}}

\begin{verbatim}
A Help to interpret modules output
23 modules
\end{verbatim}

\hypertarget{module-chipsec.modules.common.spi_fdopss}{%
\subsection{Module
chipsec.modules.common.spi\_fdopss}\label{module-chipsec.modules.common.spi_fdopss}}

\hypertarget{description}{%
\subsubsection{Description}\label{description}}

This module checks for SPI Controller Flash Descriptor Security Override
Pin Strap (FDOPSS). On some systems, this may be routed to a jumper on
the motherboard and allow to override Flash Descriptor Security or to
enable Intel ME Debug mode

\hypertarget{technical-informations}{%
\subsubsection{Technical informations}\label{technical-informations}}

\begin{itemize}
\tightlist
\item
  HSFS{[}FLOCKDN{]} is not checked but in
  chipsec.modules.common.spi\_lock module.
\item
  HSFS{[}FDOPSS{]} is status bit (Read only)
\end{itemize}

\hypertarget{valid-output}{%
\subsubsection{Valid output}\label{valid-output}}

\begin{verbatim}
[x][ Module: SPI Flash Descriptor Security Override Pin-Strap
[x][ =======================================================================
HSFS = 0xE008 << Hardware Sequencing Flash Status Register (SPIBAR + 0x4)
    [00] FDONE            = 0 << Flash Cycle Done 
    [01] FCERR            = 0 << Flash Cycle Error 
    [02] AEL              = 0 << Access Error Log 
    [03] BERASE           = 1 << Block/Sector Erase Size 
    [05] SCIP             = 0 << SPI cycle in progress 
    [13] FDOPSS           = 1 << Flash Descriptor Override Pin-Strap Status 
    [14] FDV              = 1 << Flash Descriptor Valid 
    [15] FLOCKDN          = 1 << Flash Configuration Lock-Down
  
[+] PASSED: SPI Flash Descriptor Security Override is disabled
\end{verbatim}

\hypertarget{module-chipsec.modules.common.spi_desc}{%
\subsection{Module
chipsec.modules.common.spi\_desc}\label{module-chipsec.modules.common.spi_desc}}

\hypertarget{description-1}{%
\subsubsection{Description}\label{description-1}}

This module checks that software cannot write to the flash descriptor.
The SPI Flash Descriptor (Region 0 of SPI Flash) indicates read/write
permissions for devices to access regions of the flash memory.

\hypertarget{technical-informations-1}{%
\subsubsection{Technical informations}\label{technical-informations-1}}

\begin{itemize}
\tightlist
\item
  If software can write to the Flash Descriptor, then software could
  bypass any protection defined by it. While often used for debugging,
  this should not be the case on production systems.
\item
  FRAP{[}BMRAG{]} and FRAP{[}BMWAG{]} can grant one or more masters read
  access to the BIOS region 1 overriding the read permissions in the
  Flash Descriptor.
\item
  FRAP{[}BRRA{]} and FRAP{[}BRWA{]} are not checked but in
  chipsec.modules.common.spi\_access module.
\item
  if FRAP{[}FLOCKDN{]} = 1 then FRAP register is locked.
\end{itemize}

\hypertarget{valid-output-1}{%
\subsubsection{Valid output}\label{valid-output-1}}

\begin{verbatim}
[x][ Module: SPI Flash Region Access Control
[x][ =======================================================================
FRAP = 0x00000A0B << SPI Flash Regions Access Permissions Register (SPIBAR + 0x50)
    [00] BRRA             = B << BIOS Region Read Access 
    [08] BRWA             = A << BIOS Region Write Access 

    [16] BMRAG            = 0 << BIOS Master Read Access Grant 
    [24] BMWAG            = 0 << BIOS Master Write Access Grant
    
    Software access to SPI flash regions: read = 0x0B, write = 0x0A
    [+] PASSED: SPI flash permissions prevent SW from writing to flash descriptor
\end{verbatim}

\hypertarget{module-chipsec.modules.common.ia32cfg}{%
\subsection{Module
chipsec.modules.common.ia32cfg}\label{module-chipsec.modules.common.ia32cfg}}

\hypertarget{description-2}{%
\subsubsection{Description}\label{description-2}}

This module checks that IA-32/IA-64 architectural features are
configured and locked, including IA32 Model Specific Registers (MSRs).

\hypertarget{technical-informations-2}{%
\subsubsection{Technical informations}\label{technical-informations-2}}

\begin{itemize}
\tightlist
\item
  For each CPU, Bit 0 (Lock) of IA32\_Feature\_Control register defines
  if features are locked or not from software access.
\item
  CPU Register IA32\_Feature\_Control is used to control/enable/disable
  options (BIOS software uses this register) :

  \begin{itemize}
  \tightlist
  \item
    VMX (bits 1, 2) : enable extensions in or outside SMX operation, on
    each cpu
  \item
    enable SGX (code/data memory encryption and isolation, bits 17, 18)
    on each cpu
  \end{itemize}
\item
  Safer Mode Extensions (SMX) provide a programming interface for system
  software to establish a measured environment within the platform to
  support trust decisions by end users. SMX functionality is provided in
  an Intel 64 processor through the GETSEC instruction via leaf
  functions.
\end{itemize}

\hypertarget{valid-output-2}{%
\subsubsection{Valid output}\label{valid-output-2}}

\begin{verbatim}
[x][ Module: IA32 Feature Control Lock
[x][ =======================================================================
Verifying IA32_Feature_Control MSR is locked on all logical CPUs..
     cpu0: IA32_Feature_Control Lock = 1
     cpu1: IA32_Feature_Control Lock = 1
     cpu2: IA32_Feature_Control Lock = 1
     cpu3: IA32_Feature_Control Lock = 1
     cpu4: IA32_Feature_Control Lock = 1
     cpu5: IA32_Feature_Control Lock = 1
     cpu6: IA32_Feature_Control Lock = 1
     cpu7: IA32_Feature_Control Lock = 1

    [+] PASSED: IA32_FEATURE_CONTROL MSR is locked on all logical CPUs
\end{verbatim}

\hypertarget{module-chipsec.modules.common.spi_lock}{%
\subsection{Module
chipsec.modules.common.spi\_lock}\label{module-chipsec.modules.common.spi_lock}}

\hypertarget{description-3}{%
\subsubsection{Description}\label{description-3}}

This module checks that the SPI Flash Controller configuration is
locked.

\hypertarget{technical-informations-3}{%
\subsubsection{Technical informations}\label{technical-informations-3}}

\begin{itemize}
\tightlist
\item
  The configuration of the SPI controller, including protected ranges
  (PR0-PR4), is locked by HSFS{[}FLOCKDN{]} until reset. If not locked,
  the controller configuration may be modify by reprogramming these
  registers.
\item
  Only HSFS{[}FLOCKDN{]} is checked.
\end{itemize}

\hypertarget{valid-output-3}{%
\subsubsection{Valid output}\label{valid-output-3}}

\begin{verbatim}
[x][ Module: SPI Flash Controller Configuration Locks
 [x][ =======================================================================
HSFS = 0xE008 << Hardware Sequencing Flash Status Register (SPIBAR + 0x4)
    [00] FDONE            = 0 << Flash Cycle Done 
    [01] FCERR            = 0 << Flash Cycle Error 
    [02] AEL              = 0 << Access Error Log 
    [03] BERASE           = 1 << Block/Sector Erase Size 
    [05] SCIP             = 0 << SPI cycle in progress 
    [13] FDOPSS           = 1 << Flash Descriptor Override Pin-Strap Status 
    [14] FDV              = 1 << Flash Descriptor Valid 
    [15] FLOCKDN          = 1 << Flash Configuration Lock-Down
    SPI Flash Controller configuration is locked

    [+] PASSED: SPI Flash Controller locked correctly.
\end{verbatim}

\hypertarget{module-chipsec.modules.common.me_mfg_mode}{%
\subsection{Module
chipsec.modules.common.me\_mfg\_mode}\label{module-chipsec.modules.common.me_mfg_mode}}

\hypertarget{description-4}{%
\subsubsection{Description}\label{description-4}}

This module checks that specific ME (Management Engine) mode :
Manufacturing mode. This mode allows configuring critical platform
settings stored in one-time-programmable memory (FUSEs) and some of them
are called Field Programmable Fuses (FPFs). FPFs are typically used to
store platform parameters.

\hypertarget{technical-informations-4}{%
\subsubsection{Technical informations}\label{technical-informations-4}}

Setting FPFs requires Intel's ME to be in the Manufacturing Mode. As
part of a two-step process, the FPFs are first stored to temporary
memory and are then burned when the Manufacturing Mode is closed. If a
system remains in Manufacturing Mode, that means the FPFs have never
been initialized because the process hasn't been completed. If
manufacturers somehow forget to set the FPFs that they need to set for
their products and the Manufacturing Mode remains enabled, that could
allow attackers to set their own FPFs, and, thus, control the platform.
For instance, the attackers could set their own values for Intel
BootGuard or other security features. The Intel platform would then
automatically load with the attackers' malicious code, regardless of the
steps the user would take to protect their machine against malware

\hypertarget{exploitation-of-misconfiguration}{%
\subsubsection{Exploitation of
misconfiguration}\label{exploitation-of-misconfiguration}}

\begin{itemize}
\tightlist
\item
  \url{https://blog.ptsecurity.com/2018/10/intel-me-manufacturing-mode-macbook.html}
\end{itemize}

\hypertarget{valid-output-4}{%
\subsubsection{Valid output}\label{valid-output-4}}

\begin{verbatim}
[x][ Module: ME Manufacturing Mode
[x][ =======================================================================
[+] PASSED: ME is not in Manufacturing Mode
\end{verbatim}

\hypertarget{module-chipsec.modules.common.bios_ts}{%
\subsection{Module
chipsec.modules.common.bios\_ts}\label{module-chipsec.modules.common.bios_ts}}

\hypertarget{description-5}{%
\subsubsection{Description}\label{description-5}}

This module checks for BIOS Interface Lock including Top Swap Mode.
Top-Block Swap mode is used to allow for safe update of the Boot Block
even when a power failure occurs.

\hypertarget{technical-informations-5}{%
\subsubsection{Technical informations}\label{technical-informations-5}}

3 bits of 3 differents registers are checked :

\begin{itemize}
\tightlist
\item
  General Control and Status Register / GCS{[}0{]} : BIOS Interface
  Lock-Down (BILD)
\item
  BIOS Control Register BIOS\_CNTL{[}4{]} : Top Swap Status (TSS)
\item
  Backed Up Control Register / BUC{[}0{]} : Top Swap (TS)
\end{itemize}

\hypertarget{exploitation-of-misconfiguration-1}{%
\subsubsection{Exploitation of
misconfiguration}\label{exploitation-of-misconfiguration-1}}

\begin{itemize}
\tightlist
\item
  BIOS Boot Hijacking And VMware Vulnerabilities Digging :
  \url{http://powerofcommunity.net/poc2007/sunbing.pdf}
\end{itemize}

\hypertarget{valid-output-5}{%
\subsubsection{Valid output}\label{valid-output-5}}

\begin{verbatim}

[x][ Module: BIOS Interface Lock (including Top Swap Mode)
[x][ =======================================================================
  [*] BiosInterfaceLockDown (BILD) control = 1
  [*] BIOS Top Swap mode is disabled (TSS = 0)
  [*] RTC TopSwap control (TS) = 0
  [+] PASSED: BIOS Interface is locked (including Top Swap Mode)
\end{verbatim}

\hypertarget{module-chipsec.modules.common.smrr}{%
\subsection{Module
chipsec.modules.common.smrr}\label{module-chipsec.modules.common.smrr}}

\hypertarget{description-6}{%
\subsubsection{Description}\label{description-6}}

This module checks to see that SMRRs are enabled and configured.

\hypertarget{technical-informations-6}{%
\subsubsection{Technical informations}\label{technical-informations-6}}

\begin{itemize}
\tightlist
\item
  If ring 0 software can make SMRAM cacheable and then populate cache
  lines at SMBASE with exploit code, then when an SMI is triggered, the
  CPU could execute the exploit code from cache.
\item
  To avoid this attack, System Management Mode Range Registers (SMRRs)
  force non-cachable behavior and block access to SMRAM when the CPU is
  not in SMM. These registers need to be enabled/configured by the BIOS
  and define the protected range of memory and the type of memory for
  NVRAM (cacheable or not).
\end{itemize}

\hypertarget{exploitation-of-misconfiguration-2}{%
\subsubsection{Exploitation of
misconfiguration}\label{exploitation-of-misconfiguration-2}}

Researchers demonstrated a way to use CPU cache to effectively change
values in SMRAM in :

\begin{itemize}
\tightlist
\item
  Attacking SMM Memory via Intel CPU Cache Poisoning :
  \url{http://www.invisiblethingslab.com/resources/misc09/smm_cache_fun.pdf}
\item
  Getting into the SMRAM: SMM Reloaded :
  \url{http://cansecwest.com/csw09/csw09-duflot.pdf}
\end{itemize}

\hypertarget{valid-output-6}{%
\subsubsection{Valid output}\label{valid-output-6}}

\begin{verbatim}
[x][ Module: CPU SMM Cache Poisoning / System Management Range Registers
[x][ =======================================================================
  [+] OK. SMRR range protection is supported

  [*] Checking SMRR range base programming..
  [*] IA32_SMRR_PHYSBASE = 0xCF800006 << SMRR Base Address MSR (MSR 0x1F2)
      [00] Type             = 6 << SMRR memory type 
      [12] PhysBase         = CF800 << SMRR physical base address 
  [*] SMRR range base: 0x00000000CF800000
  [*] SMRR range memory type is Writeback (WB)
  [+] OK so far. SMRR range base is programmed

  [*] Checking SMRR range mask programming..
  [*] IA32_SMRR_PHYSMASK = 0xFF800800 << SMRR Range Mask MSR (MSR 0x1F3)
      [11] Valid            = 1 << SMRR valid 
      [12] PhysMask         = FF800 << SMRR address range mask 
  [*] SMRR range mask: 0x00000000FF800000
  [+] OK so far. SMRR range is enabled

  [*] Verifying that SMRR range base & mask are the same on all logical CPUs..
  [CPU0] SMRR_PHYSBASE = 00000000CF800006, SMRR_PHYSMASK = 00000000FF800800
  [CPU1] SMRR_PHYSBASE = 00000000CF800006, SMRR_PHYSMASK = 00000000FF800800
  [CPU2] SMRR_PHYSBASE = 00000000CF800006, SMRR_PHYSMASK = 00000000FF800800
  [CPU3] SMRR_PHYSBASE = 00000000CF800006, SMRR_PHYSMASK = 00000000FF800800
  [CPU4] SMRR_PHYSBASE = 00000000CF800006, SMRR_PHYSMASK = 00000000FF800800
  [CPU5] SMRR_PHYSBASE = 00000000CF800006, SMRR_PHYSMASK = 00000000FF800800
  [CPU6] SMRR_PHYSBASE = 00000000CF800006, SMRR_PHYSMASK = 00000000FF800800
  [CPU7] SMRR_PHYSBASE = 00000000CF800006, SMRR_PHYSMASK = 00000000FF800800
  [+] OK so far. SMRR range base/mask match on all logical CPUs
  [*] Trying to read memory at SMRR base 0xCF800000..

  [+] PASSED: SMRR reads are blocked in non-SMM mode
  [+] PASSED: SMRR protection against cache attack is properly configured
\end{verbatim}

\hypertarget{module-chipsec.modules.common.smm}{%
\subsection{Module
chipsec.modules.common.smm}\label{module-chipsec.modules.common.smm}}

\hypertarget{description-7}{%
\subsubsection{Description}\label{description-7}}

This This module simply reads SMRAMC and checks that D\_LCK is set.

\hypertarget{technical-informations-7}{%
\subsubsection{Technical informations}\label{technical-informations-7}}

If SMRAMC{[}D\_LCK{]} is not set by the BIOS, SMRAM can be accessed even
when the CPU is not in SMM. SMRACMC{[}D\_LCK{]} allow to set in read
only mode for several important bits of SMRAM registers (D\_OPEN,
G\_SMRARE, C\_BASE\_SEG, H\_SMRAM\_EN, GMS, TOLUD, TOM, TSEG\_SG,
TSEG\_EN)

\hypertarget{exploitation-of-misconfiguration-3}{%
\subsubsection{Exploitation of
misconfiguration}\label{exploitation-of-misconfiguration-3}}

\begin{itemize}
\tightlist
\item
  Using CPU SMM to Circumvent OS Security Functions :
  \url{http://fawlty.cs.usfca.edu/~cruse/cs630f06/duflot.pdf}
\item
  Using SMM for Other Purposes :
  \url{http://phrack.org/issues/65/7.html}
\end{itemize}

\hypertarget{valid-output-7}{%
\subsubsection{Valid output}\label{valid-output-7}}

\begin{verbatim}
[x][ Module: Compatible SMM memory (SMRAM) Protection
[x][ =======================================================================
  [*] PCI0.0.0_SMRAMC = 0x1A << System Management RAM Control (b:d.f 00:00.0 + 0x88)
    [00] C_BASE_SEG       = ? << SMRAM Base Segment = ?b 
    [03] G_SMRAME         = 1 << SMRAM Enabled 
    [04] D_LCK            = 1 << SMRAM Locked 
    [05] D_CLS            = 0 << SMRAM Closed 
    [06] D_OPEN           = 0 << SMRAM Open 
    [*] Compatible SMRAM is enabled
    
    [+] PASSED: Compatible SMRAM is locked down
\end{verbatim}

\hypertarget{module-chipsec.modules.common.memlock}{%
\subsection{Module
chipsec.modules.common.memlock}\label{module-chipsec.modules.common.memlock}}

\hypertarget{description-8}{%
\subsubsection{Description}\label{description-8}}

This module checks if memory configuration is locked to protect SMM

\hypertarget{valid-output-8}{%
\subsubsection{Valid output}\label{valid-output-8}}

\begin{verbatim}
[x][ Module: Check MSR_LT_LOCK_MEMORY
[x][ =======================================================================
  [X] Checking MSR_LT_LOCK_MEMORY status
  [*]   cpu0: MSR_LT_LOCK_MEMORY[LT_LOCK] = 1
  [*]   cpu1: MSR_LT_LOCK_MEMORY[LT_LOCK] = 1
  [*]   cpu2: MSR_LT_LOCK_MEMORY[LT_LOCK] = 1
  [*]   cpu3: MSR_LT_LOCK_MEMORY[LT_LOCK] = 1
  [*]   cpu4: MSR_LT_LOCK_MEMORY[LT_LOCK] = 1
  [*]   cpu5: MSR_LT_LOCK_MEMORY[LT_LOCK] = 1
  [*]   cpu6: MSR_LT_LOCK_MEMORY[LT_LOCK] = 1
  [*]   cpu7: MSR_LT_LOCK_MEMORY[LT_LOCK] = 1

  [+] PASSED: Check have successfully
\end{verbatim}

\hypertarget{module-chipsec.modules.common.rtclock}{%
\subsection{Module
chipsec.modules.common.rtclock}\label{module-chipsec.modules.common.rtclock}}

\hypertarget{description-9}{%
\subsubsection{Description}\label{description-9}}

This module checks for RTC memory locks. RTC stands for Real Time Clock,
which is the crystal oscillator controlled timer that maintains the time
and date in the computer when switched off. Today, most computers have
moved the settings from CMOS and integrated them into the southbridge or
Super I/O chips.

\hypertarget{technical-informations-8}{%
\subsubsection{Technical informations}\label{technical-informations-8}}

Since we do not know what RTC memory will be used for on a specific
platform, WARNING (rather than FAILED) is returned if the memory is not
locked.

\hypertarget{valid-output-9}{%
\subsubsection{Valid output}\label{valid-output-9}}

\begin{verbatim}
[x][ Module: Protected RTC memory locations
[x][ =======================================================================
  [*] RC = 0x0000001C << RTC Configuration (RCBA + 0x3400)
    [02] UE               = 1 << Upper 128 Byte Enable 
    [03] LL               = 1 << Lower 128 Byte Lock 
    [04] UL               = 1 << Upper 128 Byte Lock 
  [+] Protected bytes (0x38-0x3F) in low 128-byte bank of RTC memory are locked
  [+] Protected bytes (0x38-0x3F) in high 128-byte bank of RTC memory are locked

  [+] PASSED: Protected locations in RTC memory are locked
\end{verbatim}

\hypertarget{module-chipsec.modules.remap}{%
\subsection{Module
chipsec.modules.remap}\label{module-chipsec.modules.remap}}

\hypertarget{description-10}{%
\subsubsection{Description}\label{description-10}}

This module checks the Memory Remapping Configuration (if correct and
locked).

\hypertarget{technical-informations-9}{%
\subsubsection{Technical informations}\label{technical-informations-9}}

\begin{itemize}
\tightlist
\item
  RAM Remapping must respect the condition : REMAPBASE \textless=
  REMAPLIMIT \textless{} TOUU
\item
  Protection of registers TOUUD (Top of Upper Usable DRAM), TOLUD (Top
  of Lower Usable DRAM), REMAPBASE and REMAPLIMIT (Memory Remap Base
  Address) and REMAPLIMIT (Memory Remap Limit Address) are also checked
  by chipsec.modules.memconfig module.
\end{itemize}

\hypertarget{valid-output-10}{%
\subsubsection{Valid output}\label{valid-output-10}}

\begin{verbatim}
[x][ Module: Memory Remapping Configuration
[x][ =======================================================================
[*] Registers:
[*]   TOUUD     : 0x000000026F600001
[*]   REMAPLIMIT: 0x000000026F500001
[*]   REMAPBASE : 0x00000001FF000001
[*]   TOLUD     : 0x8FA00001
[*]   TSEGMB    : 0x8B000001

[*] Memory Map:
[*]   Top Of Upper Memory: 0x000000026F600000
[*]   Remap Limit Address: 0x000000026F5FFFFF
[*]   Remap Base Address : 0x00000001FF000000
[*]   4GB                : 0x0000000100000000
[*]   Top Of Low Memory  : 0x000000008FA00000
[*]   TSEG (SMRAM) Base  : 0x000000008B000000

[*] checking memory remap configuration..
[*]   Memory Remap is enabled
[+]   Remap window configuration is correct: REMAPBASE <= REMAPLIMIT < TOUUD
[+]   All addresses are 1MB aligned
[*] checking if memory remap configuration is locked..
[+]   TOUUD is locked
[+]   TOLUD is locked
[+]   REMAPBASE and REMAPLIMIT are locked
[+] PASSED: Memory Remap is configured correctly and locked
\end{verbatim}

\hypertarget{module-chipsec.modules.smm_dma}{%
\subsection{Module
chipsec.modules.smm\_dma}\label{module-chipsec.modules.smm_dma}}

\hypertarget{description-11}{%
\subsubsection{Description}\label{description-11}}

This module examines the configuration and locking of SMRAM range
configuration protecting from DMA attacks. If it fails, then DMA
protection (TSEG) may not be securely configured to protect SMRAM.

\hypertarget{technical-informations-10}{%
\subsubsection{Technical informations}\label{technical-informations-10}}

\begin{itemize}
\tightlist
\item
  Just like SMRAM needs to be protected from software executing on the
  CPU, it also needs to be protected from devices that have direct
  access to DRAM (DMA). Protection from DMA is configured through proper
  programming of SMRAM memory range. If BIOS does not correctly
  configure and lock the configuration, then malware could reprogram
  configuration and open SMRAM area to DMA access, allowing manipulation
  of memory that should have been protected.
\item
  TSEG is an SMRAM extension and define a memory addresses range to
  protect by the motherboard chipset (TSEG Memory Base Register for
  range memory, bits 31:20). All SMRAM memory must be a part of TSEG in
  order to be protected. The TSEG configuration must be protected
  against writing (TSEG Memory Base Register for range memory, LOCK
  \textless=\textgreater{} bit 0)
\end{itemize}

\hypertarget{valid-output-11}{%
\subsubsection{Valid output}\label{valid-output-11}}

\begin{verbatim}
[x][ Module: SMM TSEG Range Configuration Check
[x][ [x][ =======================================================================
[*] TSEG      : 0x00000000CF800000 - 0x00000000CFFFFFFF (size = 0x00800000)
[*] SMRR range: 0x00000000CF800000 - 0x00000000CFFFFFFF (size = 0x00800000)

[*] checking TSEG range configuration..
  [+] TSEG range covers entire SMRAM
  [+] TSEG range is locked

  [+] PASSED: TSEG is properly configured. SMRAM is protected from DMA attacks
\end{verbatim}

\hypertarget{module-chipsec.modules.memconfig}{%
\subsection{Module
chipsec.modules.memconfig}\label{module-chipsec.modules.memconfig}}

\hypertarget{description-12}{%
\subsubsection{Description}\label{description-12}}

This module verifies memory map secure configuration, i.e.~that memory
map registers are correctly configured and locked down.

\hypertarget{technical-informations-11}{%
\subsubsection{Technical informations}\label{technical-informations-11}}

Bit LOCK (0) of each register is checked.

\hypertarget{valid-output-12}{%
\subsubsection{Valid output}\label{valid-output-12}}

\begin{verbatim}
[x][ Module: Host Bridge Memory Map Locks
[x][ =======================================================================
[+] PCI0.0.0_BDSM        = 0x00000000D0000001 - LOCKED   - Base of Graphics Stolen Memory
[+] PCI0.0.0_BGSM        = 0x00000000D0000001 - LOCKED   - Base of GTT Stolen Memory
[+] PCI0.0.0_DPR         = 0x00000000CF800001 - LOCKED   - DMA Protected Range
[+] PCI0.0.0_GGC         = 0x0000000000000003 - LOCKED   - Graphics Control
[+] PCI0.0.0_MESEG_MASK  = 0x0000007FFE000C00 - LOCKED   - Manageability Engine Limit Address Register
[+] PCI0.0.0_PAVPC       = 0x0000000000000004 - LOCKED   - PAVP Configuration
[+] PCI0.0.0_REMAPBASE   = 0x00000007FE000001 - LOCKED   - Memory Remap Base Address
[+] PCI0.0.0_REMAPLIMIT  = 0x000000082DF00001 - LOCKED   - Memory Remap Limit Address
[+] PCI0.0.0_TOLUD       = 0x00000000D0000001 - LOCKED   - Top of Low Usable DRAM
[+] PCI0.0.0_TOM         = 0x0000000800000001 - LOCKED   - Top of Memory
[+] PCI0.0.0_TOUUD       = 0x000000082E000001 - LOCKED   - Top of Upper Usable DRAM
[+] PCI0.0.0_TSEGMB      = 0x00000000CF800001 - LOCKED   - TSEG Memory Base
[+] PASSED: All memory map registers seem to be locked down
\end{verbatim}

\hypertarget{module-chipsec.modules.common.bios_wp}{%
\subsection{Module
chipsec.modules.common.bios\_wp}\label{module-chipsec.modules.common.bios_wp}}

\hypertarget{description-13}{%
\subsubsection{Description}\label{description-13}}

This module common.bios\_wp will fail if SMM-based protection is not
correctly configured and SPI protected ranges (PR registers) do not
protect the entire BIOS region.

\hypertarget{technical-informations-12}{%
\subsubsection{Technical informations}\label{technical-informations-12}}

\begin{itemize}
\tightlist
\item
  BIOSWE allow to lock writing on Flash SPI Region, BLE to control (SMI
  interruption) modifications on bit BIOSWE, SMM\_BWP prevent writing
  from kernel space.
\item
  PR0\textless=\textgreater Desc BIOS Region,
  PR1\textless=\textgreater BIOS Region, PR2\textless=\textgreater ME
  Region, PR3\textless=\textgreater GbE Region,
  PR4\textless=\textgreater PDR Region
\end{itemize}

\hypertarget{valid-output-13}{%
\subsubsection{Valid output}\label{valid-output-13}}

\begin{verbatim}
[x][ Module: BIOS Region Write Protection
[x][ =======================================================================
[*] BC = 0x00 << BIOS Control (b:d.f 00:31.0 + 0xDC)
    [00] BIOSWE           = 0 << BIOS Write Enable 
    [01] BLE              = 1 << BIOS Lock Enable 
    [02] SRC              = 0 << SPI Read Configuration 
    [04] TSS              = 0 << Top Swap Status 
    [05] SMM_BWP          = 1 << SMM BIOS Write Protection 
  [+] BIOS region write protection is enabled (writes restricted to SMM)

    [*] BIOS Region: Base = 0x????????, Limit = 0x????????
    SPI Protected Ranges
    ------------------------------------------------------------
    PRx (offset) | Value    | Base     | Limit    | WP? | RP?
    ------------------------------------------------------------
    PR0 (??)    | ???????? | ???????? | ???????? | 1   | 0 
    PR1 (??)    | ???????? | ???????? | ???????? | 1   | 0 
    PR2 (??)    | ???????? | ???????? | ???????? | 1   | 0 
    PR3 (??)    | 00000000 | 00000000 | 00000000 | 0   | 0 
    PR4 (??)    | 00000000 | 00000000 | 00000000 | 0   | 0

  [+] PASSED: BIOS is write protected (by SMM and SPI Protected Ranges)
\end{verbatim}

\hypertarget{module-chipsec.modules.common.bios_smi}{%
\subsection{Module
chipsec.modules.common.bios\_smi}\label{module-chipsec.modules.common.bios_smi}}

\hypertarget{description-14}{%
\subsubsection{Description}\label{description-14}}

This module checks that SMI events configuration is locked. SMI events
allow to stop the attempts to modify register like BIOS\_CNTL with bit 0
(BIOSWE).

\hypertarget{technical-informations-13}{%
\subsubsection{Technical informations}\label{technical-informations-13}}

\begin{itemize}
\item
  Differents registers are checked:

  \begin{itemize}
  \tightlist
  \item
    SMI\_EN register provides a lot of control over the generation of
    SMI\#.
  \item
    SMI\_LOCK bit of GEN\_PMCON\_1 register allows to lock configuration
    of SMI\_EN register.
  \item
    TCO\_LOCK bit of TCO1\_CNT register allows to lock modification of
    SMI\_EN{[}TCO\_EN{]}.
  \end{itemize}
\item
  The acronym TCO (Total Cost of Ownership) refer to a logic block in
  the Intel ICH products family.
\item
  If the SMI configuration space is not locked in writing, these events
  can be modified or deleted to allow a modification of register
  BIOS\_CNTL.
\end{itemize}

\hypertarget{exploitation-of-misconfiguration-secureboot-by-pass}{%
\subsubsection{Exploitation of misconfiguration (SecureBoot by
pass)}\label{exploitation-of-misconfiguration-secureboot-by-pass}}

\begin{itemize}
\tightlist
\item
  Summary of Attacks Against BIOS and Secure Boot :
  \url{https://www.defcon.org/images/defcon-22/dc-22-presentations/Bulygin-Bazhaniul-Furtak-Loucaides/DEFCON-22-Bulygin-Bazhaniul-Furtak-Loucaides-Summary-of-attacks-against-BIOS-UPDATED.pdf}
\end{itemize}

\hypertarget{valid-output-14}{%
\subsubsection{Valid output}\label{valid-output-14}}

\begin{verbatim}
[x][ Module: SMI Events Configuration
[x][ =======================================================================
[*] Checking SMI enables..
    Global SMI enable: 1
    TCO SMI enable   : 1
    [+] All required SMI events are enabled
    [*] Checking SMI configuration locks..
      [+] TCO SMI configuration is locked (TCO SMI Lock)
      [+] SMI events global configuration is locked (SMI Lock)

    [+] PASSED: All required SMI sources seem to be enabled and locked
\end{verbatim}

\hypertarget{module-chipsec.modules.common.bios_kbrd_buffer}{%
\subsection{Module
chipsec.modules.common.bios\_kbrd\_buffer}\label{module-chipsec.modules.common.bios_kbrd_buffer}}

\hypertarget{description-15}{%
\subsubsection{Description}\label{description-15}}

This module checks for BIOS/HDD password exposure through BIOS keyboard
buffer and checks for exposure of pre-boot passwords (BIOS/HDD/pre-boot
authentication SW) in the BIOS keyboard buffer.

\hypertarget{technical-informations-14}{%
\subsubsection{Technical informations}\label{technical-informations-14}}

The BIOS API offers interruption 0x16to retrieve keystrokes from the
keyboard and uses a buffer to work in order to use extended keystrokes
(e.g.: Alt + Shift + Keystroke). This Buffer, into the memory physical,
is not flushed after using and can be read by an attacker (from OS).

\hypertarget{exploitation-of-the-vulnerability}{%
\subsubsection{Exploitation of the
vulnerability}\label{exploitation-of-the-vulnerability}}

\begin{itemize}
\tightlist
\item
  Bypassing Pre-boot Authentication Passwords by Instrumenting the BIOS
  Keyboard Buffer :
  \url{https://www.defcon.org/images/defcon-16/dc16-presentations/brossard/defcon-16-brossard-wp.pdf}
\end{itemize}

\hypertarget{valid-output-15}{%
\subsubsection{Valid output}\label{valid-output-15}}

\begin{verbatim}
[x][ Module: Pre-boot Passwords in the BIOS Keyboard Buffer
[x][ =======================================================================
[*] Keyboard buffer head pointer = 0x0 (at 0x41A), tail pointer = 0x0 (at 0x41C)
[*] Keyboard buffer contents (at 0x41E):
00 00 00 00 00 00 00 00 00 00 00 00 00 00 00 00 |                 
00 00 00 00 00 00 00 00 00 00 00 00 00 00 00 00 |                 
[*] Checking contents of the keyboard buffer..

[+] PASSED: Keyboard buffer looks empty. Pre-boot passwords don't seem to be exposed
\end{verbatim}

\hypertarget{module-chipsec.modules.common.spi_access}{%
\subsection{Module
chipsec.modules.common.spi\_access}\label{module-chipsec.modules.common.spi_access}}

\hypertarget{description-16}{%
\subsubsection{Description}\label{description-16}}

This module checks the SPI Flash Region Access Permissions programmed in
the Flash Descriptor from Software access.

\hypertarget{technical-informations-15}{%
\subsubsection{Technical informations}\label{technical-informations-15}}

\begin{itemize}
\tightlist
\item
  If the bit is set for bytes HSFS{[}BRRA{]} and HSFS{[}BRWA{]}, the
  specific master can erase and write that particular region through
  register accesses
\item
  HSFS{[}BMRAG{]} and HSFS{[}BMWAG{]} are not checked but in
  chipsec.modules.common.spi\_desc module
\item
  if HSFS{[}FLOCKDN{]} = 1 then HSFS{[}FRAP{]} register is locked
\item
  BRRA and BRWA are status bits (Read only) to indicate access rights on
  Flash Regions (7)
\item
  Read Access is accepted and write access on FREG1\_BIOS is accepted
\end{itemize}

\hypertarget{valid-output-16}{%
\subsubsection{Valid output}\label{valid-output-16}}

\begin{verbatim}
[x][ Module: SPI Flash Region Access Control
[x][ =======================================================================
SPI Flash Region Access Permissions
------------------------------------------------------------
[*] FRAP = 0x00000A0B << SPI Flash Regions Access Permissions Register (SPIBAR + 0x50)
    [00] BRRA             = 03|07|0B|0F << BIOS Region Read Access 
    [08] BRWA             = 02 << BIOS Region Write Access 
    [16] BMRAG            = 0 << BIOS Master Read Access Grant 
    [24] BMWAG            = 0 << BIOS Master Write Access Grant 

BIOS Region Write Access Grant (00):
  FREG0_FLASHD: 0
  FREG1_BIOS  : 0
  FREG2_ME    : 0
  FREG3_GBE   : 0
  FREG4_PD    : 0
  FREG5       : 0
  FREG6       : 0
BIOS Region Read Access Grant (00):
  FREG0_FLASHD: 0
  FREG1_BIOS  : 0
  FREG2_ME    : 0
  FREG3_GBE   : 0
  FREG4_PD    : 0
  FREG5       : 0
  FREG6       : 0
BIOS Region Write Access (02):
  FREG0_FLASHD: 0
  FREG1_BIOS  : 1
  FREG2_ME    : 0
  FREG3_GBE   : 0
  FREG4_PD    : 0
  FREG5       : 0
  FREG6       : 0
BIOS Region Read Access (03|07|0B|0F):
  FREG0_FLASHD: 1
  FREG1_BIOS  : 1
  FREG2_ME    : ?
  FREG3_GBE   : ?
  FREG4_PD    : 0
  FREG5       : 0
  FREG6       : 0

    [+] PASSED: SPI Flash Region Access Permissions in flash descriptor look ok
\end{verbatim}

\hypertarget{module-chipsec.modules.common.cpu.spectre_v2}{%
\subsection{Module
chipsec.modules.common.cpu.spectre\_v2}\label{module-chipsec.modules.common.cpu.spectre_v2}}

\hypertarget{description-17}{%
\subsubsection{Description}\label{description-17}}

In 2018, reasearchers discovered that CPU data cache timing can be
abused to efficiently leak information out of mis-speculated execution,
leading to arbitrary virtual memory read vulnerabilities across local
security boundaries in various contexts.

There are three known variants of the issue:

\begin{itemize}
\tightlist
\item
  Variant 1: bounds check bypass (CVE-2017-5753)
\item
  Variant 2: branch target injection (CVE-2017-5715)
\item
  Variant 3: rogue data cache load (CVE-2017-5754)
\end{itemize}

This module checks the variant 2 in verifying if system includes
hardware mitigations for Speculative Execution Side Channel.

\hypertarget{technical-informations-16}{%
\subsubsection{Technical informations}\label{technical-informations-16}}

The module checks if the following hardware mitigations (with hardware
registers) are supported by the CPU and enabled by the OS/software:

\begin{itemize}
\tightlist
\item
  Indirect Branch Restricted Speculation (IBRS) and Indirect Branch
  Predictor Barrier (IBPB):

  \begin{itemize}
  \tightlist
  \item
    CPUID.(EAX=7H,ECX=0):EDX{[}26{]} == 1 \textgreater{} enumerates
    support for IBRS and IBPB
  \end{itemize}
\item
  Single Thread Indirect Branch Predictors (STIBP):

  \begin{itemize}
  \tightlist
  \item
    CPUID.(EAX=7H,ECX=0):EDX{[}27{]} == 1 \textgreater{} enumerates
    support for STIBP
  \item
    IA32\_SPEC\_CTRL{[}STIBP{]} == 1 \textgreater{} enable control for
    STIBP by the software/OS
  \end{itemize}
\item
  Enhanced IBRS:

  \begin{itemize}
  \tightlist
  \item
    CPUID.(EAX=7H,ECX=0):EDX{[}29{]} == 1 \textgreater{} enumerates
    support for the IA32\_ARCH\_CAPABILITIES MSR
  \item
    IA32\_ARCH\_CAPABILITIES{[}IBRS\_ALL{]} == 1 \textgreater{}
    enumerates support for enhanced IBRS
  \item
    IA32\_SPEC\_CTRL{[}IBRS{]} == 1 \textgreater{} enable control for
    enhanced IBRS by the software/OS
  \end{itemize}
\end{itemize}

\hypertarget{exploitation-of-misconfiguration-4}{%
\subsubsection{Exploitation of
misconfiguration}\label{exploitation-of-misconfiguration-4}}

\begin{itemize}
\tightlist
\item
  Technical details of Spectre/Meldown :
  \url{https://googleprojectzero.blogspot.com/2018/01/reading-privileged-memory-with-side.html}
\item
  Spectre Attacks (variant 1 and 2), Exploiting Speculative Execution :
  \url{https://spectreattack.com/spectre.pdf}
\item
  Meltdown (variant 3), Reading Kernel Memory from User Space :
  \url{https://meltdownattack.com/meltdown.pdf}
\end{itemize}

\hypertarget{valid-output-17}{%
\subsubsection{Valid output}\label{valid-output-17}}

\begin{verbatim}
[x][ Module: Checks for Branch Target Injection / Spectre v2 (CVE-2017-5715)
[x][ =======================================================================
  [*] CPUID.7H:EDX[26] = 1 Indirect Branch Restricted Speculation (IBRS) & Predictor Barrier (IBPB)
  [*] CPUID.7H:EDX[27] = 1 Single Thread Indirect Branch Predictors (STIBP)
  [*] CPUID.7H:EDX[29] = 0 IA32_ARCH_CAPABILITIES
  [+] CPU supports IBRS and IBPB
  [+] CPU supports STIBP
  [*] checking enhanced IBRS support in IA32_ARCH_CAPABILITIES...
  [+] CPU supports enhanced IBRS (on all logical CPU)
  [+] OS enabled Enhanced IBRS (on all logical processors)

  [+] PASSED: CPU and OS support hardware mitigations
\end{verbatim}

\hypertarget{module-chipsec.modules.common.uefi.access_uefispec}{%
\subsection{Module
chipsec.modules.common.uefi.access\_uefispec}\label{module-chipsec.modules.common.uefi.access_uefispec}}

\hypertarget{description-18}{%
\subsubsection{Description}\label{description-18}}

This module checks protection of UEFI variables defined in the UEFI spec
to have certain permissions. Returns failure if variable attributes are
not as defined in table 11 Global Variables \url{http://uefi.org} of the
UEFI spec.

\hypertarget{technical-informations-17}{%
\subsubsection{Technical informations}\label{technical-informations-17}}

Possibilities for attributes are :

\begin{itemize}
\tightlist
\item
  Non-Volatile (NV) : Stored in SPI Flash based NVRAM
\item
  Boot Service (BS) : Accessible to DXE drivers / Boot Loaders at boot
  time
\item
  Run-Time (RT) : Accessible to the OS through run-time UEFI
  SetVariable/GetVariable API
\item
  Time-Based Authenticated Write Access (TBAWS) : see
  chipsec.modules.common.secureboot.variables section
\item
  Authenticated Write Access (AWS) : see
  chipsec.modules.common.secureboot.variables section
\end{itemize}

\hypertarget{valid-output-18}{%
\subsubsection{Valid output}\label{valid-output-18}}

\begin{verbatim}
[x][ Module: Access Control of EFI Variables
[x][ =======================================================================
[*] Testing UEFI variables ..
[*] Variable PlatformLangCodes (BS+RT)
[*] Variable BootOrder (NV+BS+RT)
[*] Variable GsetUefiIplDefaultValue (NV+BS+RT)
[*] Variable BootFlow (BS+RT)
[*] Variable Ar00000000 (NV+BS+RT)
[*] Variable ConOut (NV+BS+RT)
[*] Variable DefaultLegacyDevOrder (NV+BS+RT)
[*] Variable ProgressBarPolicyVar (BS+RT)
[*] Variable NBPlatformData (BS+RT)
[*] Variable TcgMonotonicCounter (NV+BS+RT)
[*] Variable DriverHealthCount (BS+RT)
[*] Variable ServiceTag (NV+BS+RT)
[*] Variable EsataBay (NV+BS+RT)
[*] Variable AssetTag (NV+BS+RT)
[*] Variable db (NV+BS+RT+TBAWS)
[*] Variable TdtAdvancedSetupDataVar (NV+BS+RT)
[*] Variable P (NV+BS+RT)
[*] Variable TPMPERBIOSFLAGS (NV+BS+RT)
[*] Variable GsetLegacyIplDefaultValue (NV+BS+RT)
[*] Variable BootOneDevice (BS+RT)
[*] Variable UsbMassDevNum (BS+RT)
[*] Variable ColdReset (BS+RT)
[*] Variable WdtPersistentData (NV+BS+RT)
[*] Variable ScramblerBaseSeed (NV+BS+RT)
[*] Variable ConOutDev (BS+RT)
[*] Variable SerialPortsEnabledVar (BS+RT)
[*] Variable TcgPPIVarAddr (NV+BS+RT)
[*] Variable DefaultBootOrder (NV+BS+RT)
[*] Variable UsbMassDevValid (BS+RT)
[*] Variable PK (NV+BS+RT+TBAWS)
[*] Variable BootFFFD (NV+BS+RT)
[*] Variable BootFFFE (NV+BS+RT)
[*] Variable BootFFFB (NV+BS+RT)
[*] Variable BootFFFC (NV+BS+RT)
[*] Variable Boot0001 (NV+BS+RT)
[*] Variable SetupSnbPpmFeatures (NV+BS+RT)
[*] Variable NvRamSpdMap (NV+BS+RT)
[*] Variable SetupPlatformData (BS+RT)
[*] Variable OsIndications (NV+BS+RT)
[*] Variable CurrentPriority (NV+BS+RT)
[*] Variable BootOptionSupport (BS+RT)
[*] Variable MonotonicCounter (NV+BS+RT)
[*] Variable ConInDev (BS+RT)
[*] Variable S3CpuThrottle (NV+BS+RT)
[*] Variable EDIDLastData (NV+BS+RT)
[*] Variable ErrOut (BS+RT + NV)
[*] Variable TxtFeatures (BS+RT)
[*] Variable OsType (NV+BS+RT)
[*] Variable PchS3Peim (BS+RT)
[*] Variable Lang (NV+BS+RT)
[*] Variable An00000000 (NV+BS+RT)
[*] Variable PchInit (NV+BS+RT)
[*] Variable GNVS_PTR (BS+RT)
[*] Variable OsIndicationsSupported (BS+RT)
[*] Variable FPDT_Variable (NV+BS+RT)
[*] Variable BootCurrent (BS+RT)
[*] Variable Timeout (NV+BS+RT)
[*] Variable Rd00000000 (NV+BS+RT)
[*] Variable SetupDptfFeatures (NV+BS+RT)
[*] Variable SignatureSupport (BS+RT+AWS)
[*] Variable KEK (NV+BS+RT+TBAWS)
[*] Variable SetupMode (BS+RT+AWS)
[*] Variable PreviousBootServiceDataInfo (NV+BS+RT)
[*] Variable Boot0000 (NV+BS+RT)
[*] Variable ErrOutDev (BS+RT)
[*] Variable Boot0003 (NV+BS+RT)
[*] Variable Boot0005 (NV+BS+RT)
[*] Variable MemoryOverwriteRequestControl (NV+BS+RT)
[*] Variable SecureBoot (BS+RT+AWS)
[*] Variable EfiTime (NV+BS+RT)
[*] Variable DFNS (NV+BS+RT)
[*] Variable DebuggerSerialPortsEnabledVar (BS+RT)
[*] Variable ConIn (NV+BS+RT)
[*] Variable Boot0012 (NV+BS+RT)
[*] Variable Boot0011 (NV+BS+RT)
[*] Variable Boot0010 (NV+BS+RT)
[*] Variable SaPegData (NV+BS+RT)
[*] Variable InSetup (BS+RT)
[*] Variable M (NV+BS+RT)
[*] Variable MBPDataPoint (BS+RT)
[*] Variable TxtOneTouch (NV+BS+RT)
[*] Variable OemCpuData (BS+RT)
[*] Variable NBGopPlatformData (BS+RT)
[*] Variable AMITCGPPIVAR (NV+BS+RT)
[*] Variable DriverHlthEnable (BS+RT)
[*] Variable LangCodes (BS+RT)
[*] Variable Boot000D (NV+BS+RT)
[*] Variable Boot000E (NV+BS+RT)
[*] Variable Boot000F (NV+BS+RT)
[*] Variable PlatformLang (NV+BS+RT)
[*] Variable Rc00000000 (NV+BS+RT)

[+] PASSED: All Secure Boot UEFI variables are protected
\end{verbatim}

\hypertarget{module-chipsec.modules.common.uefi.s3bootscript}{%
\subsection{Module
chipsec.modules.common.uefi.s3bootscript}\label{module-chipsec.modules.common.uefi.s3bootscript}}

\hypertarget{description-19}{%
\subsubsection{Description}\label{description-19}}

This module checks protections of the S3 resume boot-script implemented
by the UEFI based firmware. UEFI Boot Script is a data structure
interpreted by UEFI firmware during S3 resume.

\hypertarget{technical-informations-18}{%
\subsubsection{Technical informations}\label{technical-informations-18}}

\begin{itemize}
\tightlist
\item
  In some cases, an attacker with ring0 privileges can alter this data
  structure. As a result, by forcing S3 suspend/resume cycle, an
  attacker can run arbitrary code on a platform that is not yet fully
  locked (BIOS\_CNTL not locked and SMRAM via DMA not locked with TSEGMB
  lock bit). The consequences include ability to overwrite the flash
  storage and take control over SMM.
\end{itemize}

\hypertarget{exploitation-of-misconfigurations}{%
\subsubsection{Exploitation of
misconfigurations}\label{exploitation-of-misconfigurations}}

\begin{itemize}
\tightlist
\item
  Attacks on UEFI Security :
  \url{https://events.ccc.de/congress/2014/Fahrplan/system/attachments/2557/original/AttacksOnUEFI_Slides.pdf}
\item
  Attacking UEFI Boot Script :
  \url{https://bromiumlabs.files.wordpress.com/2015/01/venamis_whitepaper.pdf}
\item
  Technical details to exploit vulnerability :
  \url{http://blog.cr4.sh/2015/02/exploiting-uefi-boot-script-table.html}
\end{itemize}

\hypertarget{valid-output-19}{%
\subsubsection{Valid output}\label{valid-output-19}}

\begin{verbatim}
[x][ Module: S3 Resume Boot-Script Protections
[x][ =======================================================================
  [*] SMRAM: Base = 0x00000000CF800000, Limit = 0x00000000CFFFFFFF, Size = 0x00800000
  [+] Didn't find any S3 boot-scripts in EFI variables
  [!] WARNING: S3 Boot-Script was not found. Firmware may be using other ways to store/locate it

  OR

[x][ Module: S3 Resume Boot-Script Protections
[x][ =======================================================================
  [*] SMRAM: Base = 0x00000000CF800000, Limit = 0x00000000CFFFFFFF, Size = 0x00800000
  [*] Found ? S3 boot-script(s) in EFI variables
  [*] Checking entry-points of Dispatch opcodes..
    [+] ????? > PROTECTED
    [+] ????? > PROTECTED
    [+] ????? > PROTECTED
    [+] ????? > PROTECTED
  [+] S3 boot-script is in SMRAM
\end{verbatim}

\hypertarget{module-chipsec.modules.common.secureboot.variables}{%
\subsection{Module
chipsec.modules.common.secureboot.variables}\label{module-chipsec.modules.common.secureboot.variables}}

\hypertarget{description-20}{%
\subsubsection{Description}\label{description-20}}

This module checks that all Secure Boot key/whitelist/blacklist UEFI
variables are authenticated and protected from unauthorized modification
(only from application not signed because it is possible to modify
Secureboot variables from local access). Secureboot allows to protect
against modification of critical codes used to start a computer and its
operating system (UEFI binaries, drivers, Kernbel Grub, Windows Boot
Loader, OS Boot Loader, \ldots). Without SecureBoot and Flash
protection, UEFI BootKit can be installed easly.

\hypertarget{technical-informations-19}{%
\subsubsection{Technical informations}\label{technical-informations-19}}

SecureBoot UEFI variables are stored into SPI Flash into NVRAM volume
and various key databases are used to configure SecureBoot :

\begin{itemize}
\tightlist
\item
  DB (aka, `signature database'): contains the trusted keys used for
  authenticating any applications or drivers executed in the UEFI
  environment.
\item
  DBX (aka, `forbidden signature database' or `signature database
  blacklist'): contains a set of explicitly untrusted keys and binary
  hashes. Any application or driver signed by these keys or matching
  these hashes will be blocked from execution.
\item
  KEK (key exchange keys database): contains the set of keys trusted for
  updating DB and DBX.
\item
  PK (platform key - while PK is often referred to simply as a single
  public key, it could be implemented as a database). Only updates
  signed with PK can update the KEK database.
\item
  SecureBoot: Enables/disables image signature checks.
\item
  SetupMode: SETUP\_MODE allows updating KEK/db(x), self-signed PK.
\end{itemize}

Checked attibuts for each SecureBoot variables are :

\begin{itemize}
\tightlist
\item
  Authenticated Write Access -- PK cert verifies PK/KEK update -- KEK
  verifies db/dbx update -- certdb verifies general authenticated EFI
  variable updates
\item
  Time-Based Authenticated Write Access

  \begin{itemize}
  \tightlist
  \item
    Same attributs as Authenticated Write Access +
  \item
    Signed with time-stamp (anti-replay)
  \end{itemize}
\end{itemize}

Warning message :

\begin{itemize}
\tightlist
\item
  If Secureboot are been disabled from BIOS interface (in order to boot
  on linux live to execute Chipsec), the message : Secure Boot appears
  to be disabled will be returned.
\item
  If no blacklist has been defined, the warning message : Some required
  Secure Boot variables are missing will be returned.
\end{itemize}

\hypertarget{valid-output-20}{%
\subsubsection{Valid output}\label{valid-output-20}}

\begin{verbatim}
[x][ Module: Attributes of Secure Boot EFI Variables
[x][ =======================================================================
[*] Checking protections of UEFI variable 8be4df61-93ca-11d2-aa0d-00e098032b8c:SecureBoot
[+] Variable 8be4df61-93ca-11d2-aa0d-00e098032b8c:PK is authenticated 
\(AUTHENTICATED_WRITE_ACCESS)
[*] Checking protections of UEFI variable 8be4df61-93ca-11d2-aa0d-00e098032b8c:SetupMode
[+] Variable 8be4df61-93ca-11d2-aa0d-00e098032b8c:PK is authenticated 
\(AUTHENTICATED_WRITE_ACCESS)
[*] Checking protections of UEFI variable 8be4df61-93ca-11d2-aa0d-00e098032b8c:PK
[+] Variable 8be4df61-93ca-11d2-aa0d-00e098032b8c:PK is authenticated 
\(TIME_BASED_AUTHENTICATED_WRITE_ACCESS)
[*] Checking protections of UEFI variable 8be4df61-93ca-11d2-aa0d-00e098032b8c:KEK
[+] Variable 8be4df61-93ca-11d2-aa0d-00e098032b8c:KEK is authenticated 
\(TIME_BASED_AUTHENTICATED_WRITE_ACCESS)
[*] Checking protections of UEFI variable d719b2cb-3d3a-4596-a3bc-dad00e67656f:db
[+] Variable d719b2cb-3d3a-4596-a3bc-dad00e67656f:db is authenticated \
(TIME_BASED_AUTHENTICATED_WRITE_ACCESS)
[*] Checking protections of UEFI variable d719b2cb-3d3a-4596-a3bc-dad00e67656f:dbx
[+] Variable d719b2cb-3d3a-4596-a3bc-dad00e67656f:dbx is authenticated \
(TIME_BASED_AUTHENTICATED_WRITE_ACCESS)

[+] PASSED: All Secure Boot UEFI variables are protected
\end{verbatim}

\hypertarget{module-chipsec.modules.debugenabled}{%
\subsection{Module
chipsec.modules.debugenabled}\label{module-chipsec.modules.debugenabled}}

\hypertarget{description-21}{%
\subsubsection{Description}\label{description-21}}

This module checks if the system has debug features turned on,
specifically the Direct Connect Interface (DCI).

\hypertarget{technical-informations-20}{%
\subsubsection{Technical informations}\label{technical-informations-20}}

\begin{itemize}
\tightlist
\item
  With activated DCI interface and debug options, an attacker can simply
  plug into an external USB port to install a persistent rootkit,
  bypassing secure boot and many other security features.
\item
  The module checks if the following hardware are well configurated :

  \begin{itemize}
  \tightlist
  \item
    P2SB\_DCI.DCI\_CONTROL\_REG{[}HDCIEN{]} \textgreater{} to
    enable/disable DCI Direct Connect Interface
  \item
    IA32\_DEBUG\_INTERFACE{[}DEBUGENABLE{]} \textgreater{} to
    enable/disable CPU debug features
  \item
    IA32\_DEBUG\_INTERFACE{[}DEBUGLOCK{]} \textgreater{} to unlock/lock
    manually or automatically (with SMI\#)
  \item
    IA32\_DEBUG\_INTERFACE{[}DEBUGEOCCURED{]} \textgreater{} to indicate
    the status of bit DEBUGENABLE (RO)
  \end{itemize}
\end{itemize}

\hypertarget{exploitation-of-misconfiguration-5}{%
\subsubsection{Exploitation of
misconfiguration}\label{exploitation-of-misconfiguration-5}}

\begin{itemize}
\tightlist
\item
  Evil Maid Firmware Attacks Using USB Debug :
  \url{https://eclypsium.com/2018/07/23/evil-maid-firmware-attacks-using-usb-debug/}
\end{itemize}

\hypertarget{valid-output-21}{%
\subsubsection{Valid output}\label{valid-output-21}}

\begin{verbatim}
  [*] NOT IMPLEMENTED: CPU Debug features are not supported on this platform
    Skipping module chipsec.modules.debugenabled since it is not supported in this platform

OR

[x][ Module: CPU Debug features are not supported on this platform
[x][ =======================================================================
  [X] Checking IA32_DEBUG_INTERFACE msr status
  [+] CPU IA32_DEBUG_INTERFACE is enabled
  [X] Checking DCI register status
  [+] DCI Debug is disabled

  [+] All checks have successfully passed
\end{verbatim}

\hypertarget{module-chipsec.modules.common.sgx_check}{%
\subsection{Module
chipsec.modules.common.sgx\_check}\label{module-chipsec.modules.common.sgx_check}}

\hypertarget{description-22}{%
\subsubsection{Description}\label{description-22}}

This module checks SGX related configuration into hardware register and
protected memory. SGX is a CPU option to isolate and encrypt code and
data between same process. This option consuming resources must be
skippable.

\hypertarget{technical-informations-21}{%
\subsubsection{Technical informations}\label{technical-informations-21}}

The following hardware are checked :

\begin{itemize}
\tightlist
\item
  IA32\_FEATURE\_CONTROL{[}LOCK{]} \textgreater{} to prevent writing on
  this register
\item
  IA32\_FEATURE\_CONTROL{[}ENABLE{]} \textgreater{} to enable/disable
  SGX
\end{itemize}

\hypertarget{valid-output-22}{%
\subsubsection{Valid output}\label{valid-output-22}}

\begin{verbatim}
[x][ Module: Check SGX feature support
[x][ =======================================================================
[*] Test if CPU has support for SGX
[*] SGX BIOS enablement check
[*] Verifying IA32_FEATURE_CONTROL MSR is configured
[+] Intel SGX is Enabled in BIOS

[*] Verifying IA32_FEATURE_CONTROL MSR is locked
[+] IA32_Feature_Control locked

[*] Verifying if Protected Memory Range (PRMRR) is configured
[+] Protected Memory Range configuration is supported
[*] Verifying PRMR Configuration on each core.
[+] PRMRR config is uniform across all CPUs

[*] Verifying if SGX instructions are supported
...
[+] Intel SGX is available to use
...
[*] Check SGX debug feature settings
[+] SGX debug mode is disabled

[+] All SGX checks passed

OR

[!] Intel SGX instructions disabled by firmware

OR 

[*] NOT IMPLEMENTED: CPU Debug features are not supported on this platform
Skipping module chipsec.modules.debugenabled since it is not supported in this platform
\end{verbatim}
