\documentclass{report}

\usepackage[french]{babel}
\usepackage[T1]{fontenc}
\usepackage[utf8]{inputenc}
\usepackage{hyperref}
\usepackage{tabularx}
\usepackage{microtype}
\usepackage{fullpage}

\title{Procédure de tests \\ SAD Bureautique}
\author{Agence Nationale de la Sécurité des Systèmes d’Information (ANSSI)}
\date{2018}

\begin{document}
\maketitle

\newdimen\longline
\longline=\textwidth\advance\longline-4cm

\def\LayoutTextField#1#2{#2} % override default in hyperref

\def\lbl#1{\hbox to 4cm{#1\dotfill\strut}}%
\def\labelline#1#2{\lbl{#1}\vbox{\hbox{\TextField[name=#1,width=#2]{\null}}\kern2pt\hrule}}

\def\q#1{\hbox to \hsize{\labelline{#1}{\longline}}\vskip1.4ex}

\tableofcontents

% =============================================================================
\part{Objectifs}
% =============================================================================

Le principal objectif de ce document est de fournir une procédure bien
documentée afin de vérifier, pour un matériel informatique donné, le respect ou
non par ledit matériel des exigences de sécurité du SAD Bureautique 2016--2019.

Il est important de rappeler que la vérification de conformité aux exigences de
sécurité du CCTP ne constitue en rien un audit de sécurité.

% =============================================================================
\part{Procédures}

\chapter{Exigences matérielles}
% -----------------------------------------------------------------------------

\section{Exigence 1}

\begin{quote}
  Pour chaque lot de machines livré et en fonction des options, tous les
  composants matériels doivent être listés et être identiques.
  %
  La liste contiendra les spécifications techniques complètes suivantes~:
  fabricant, modèle, version, version des logiciels embarqués (y compris
  \emph{firmware}).
\end{quote}

\paragraph{Motivation}
%
Cette exigence permet à l’autorité qualifiée de connaître précisément quels sont
les composants matériels utilisés dans les matériels informatiques qu’elle se
procure.
%
Cela comporte au moins deux avantages:

\begin{enumerate}
\item Lors de l’annonce d’une vulnérabilité critique, il devient théoriquement
  possible de déterminer quels sont les lots impactés ;
\item Une évaluation des émissions électromagnétiques d’une machine d’un lot
  vaut normalement pour le lot complet.
\end{enumerate}

\paragraph{Procédure}
Si le candidat ne fournit pas de lui-même la liste demandée, le courriel de
relance par le vérificateur doit contenir le modèle décrit en
Annexe~\ref{annexe:modele}.
%
Pour s’assurer de l’exactitude de la liste fournie, le vérificateur peut
utiliser des logiciels comme \texttt{lspci} ou \texttt{lshw}.
%

\section{Exigence 2}

\begin{quote}
  Le titulaire fournit les rapports d’évaluation suivant les normes EN 55024 et
  EN 55022 effectués pour le marquage CE selon la directive européenne DIR
  2014/30/UE et tous documents prouvant la conformité des machines de série
  suivant la directive européenne DIR 2014/30/UE annexe II chapitre 4.
\end{quote}

\paragraph{Motivation}
%
En fournissant les documents demandés, le candidat prouve que son matériel
respecte la réglementation européenne en vigueur.

\paragraph{Procédure}
%
Il suffit juste de s’assurer qu’une copie des documents demandés a effectivement
été fournie par le candidat.

\section{Exigence 3}

\begin{quote}
  Le titulaire fournit les documentations techniques de l’ensemble des
  périphériques exécutant un \emph{firmware} et susceptibles de fournir un moyen
  de communication (filaire ou non filaire).
  %
  La nature des connexions que les périphériques activés par défaut initient ou
  reçoivent doivent être précisée dans la documentation technique.
  %
  L’ensemble des options susceptibles de modifier le comportement du poste sur
  le réseau doivent être listées.
  %
  La documentation est fournie en langue française ou, à défaut, en langue
  anglaise.
\end{quote}

\paragraph{Motivation}
%
Les matériels informatiques contiennent de plus en plus de composants matériels
autonomes, pouvant notamment initier des communications réseaux avec l’extérieur
(par exemple, un serveur du candidat) sans que l’autorité qualifiée ne soit
forcément au courant.

\paragraph{Procédure}
%
Le \emph{Management Engine} (ME) est sans doute le composant le plus connu
présentant un comportement autonome important.
%
Il convient de vérifier dans la configuration du BIOS si certains de ses modules
sont présents et activés par défaut.
%
Le vérificateur pourra notamment chercher les noms suivants :

\begin{itemize}
\item \emph{Active Management Technology} (AMT)
\item \emph{Computrace}, \emph{Anti-Theft}
\end{itemize}

L’Annexe~\ref{annexe:modele} comporte une colonne qui permet d’indiquer au
candidat si un composant matériel présente un tel comportement.

\section{Exigence 4}

\begin{quote}
  Le titulaire doit être en mesure de fournir la liste complète des documents de
  spécification technique et des guides de programmation des composants
  matériels qui sont initialisés ou manipulés par le \emph{firmware}.
  %
  La liste distingue les informations disponibles publiquement de celles
  soumises à un accord de confidentialité.
  %
  Si les accords de confidentialité interdisent la transmission de ces
  informations, le nombre de périphériques concernés doit être indiqué.
  %
  La liste précise également les composants dont les descriptions techniques
  sont indisponibles ou partielles.
  %
  La langue du document es précisée pour chacun.
\end{quote}

\paragraph{Motivation}
%
L’intérêt de cette exigence est de permettre à l’autorité qualifiée de savoir
précisément composants matériels sont utilisés par le BIOS et, potentiellement,
comment elle peut elle-même en tirer parti (grâce aux documents de spécification
technique).

\paragraph{Procédure}
%
Vérifier grâce à des outils standards (\texttt{lspci}, \texttt{lshw}, etc.)  si
les composants matériels reportés concordent avec la liste fournie par le
candidat.

\section{Exigence 5}

\begin{quote}
  Si une interface matérielle n’est pas listée dans la configuration demandée,
  elle doit être soit physiquement absente de la configuration livrée, soit
  présente et désactivée par défaut.
  %
  Sa présence doit être documentée explicitement et elle ne peut être activée
  que par l’autorité qualifiée (pas par l’utilisateur) de façon sécurisée.
  %
  La désactivation de cette interface ne doit pas être plus complexe que son
  activation; les prérequis nécessaires à l’activation doivent être suffisants
  pour la désactivation. Si les modules sont présents et désactivés, ils ne
  doivent ni initier ni accepter des connexions réseau.
\end{quote}

\paragraph{Motivation}
%
Afin de ne pas augmenter inutilement la surface d’attaque potentiellement
exploitable par un attaquant, désactiver les interfaces de communication non
nécessaires (3G, \emph{Blutooth}) est une bonne pratique.
%
Le CCTP décrit précisément les configurations matérielles attendues.
%
Quand un candidat propose un matériel qui dispose d’options supplémentaires par
rapport à ces configurations, il est important que ces dernières soient
désactivées par défaut.

\paragraph{Procédure}
%
Une revue détaillée des options de configuration de BIOS est importante pour
s’assurer de l'état des interfaces de configuration supplémentaires.
%
Il peut aussi être intéressant de s’assurer, grâce à des outils standards comme
\texttt{lshw} ou \texttt{lspci}, du respect par le BIOS de cette configuration.

\section{Exigence 6}

\begin{quote}
  Les interfaces de communication sans fil devront être soit absente, soit
  démontables physiquement (par l’autorité qualifiée) sans perturber le bon
  fonctionnement de la machine et sans influence sur la garantie.
\end{quote}

\paragraph{Motivation}
%
Cette exigence suit la même philosophie que la précédente : limiter au besoin la
surface d’attaque à disposition pour l’attaquant.

\paragraph{Procédure}
%
Le vérificateur peut essayer de démonter lui-même les interfaces de
communication concernées.

\section{Exigence 7}

\begin{quote}
  La plate-forme dispose d’un système de virtualisation des entrées sorties
  (I/OMMU), activable dans l’interface de configuration du firmware.
\end{quote}

\paragraph{Motivation}
%
La virtualisation des entrées sorties est une technologie relativement mature
qui peut permettre, de manière transparente pour l’utilisateur final, de limiter
l’impact de la compromission d’un périphérique matériel.

\paragraph{Procédure}
%
L’I/OMMU d’Intel s’appelle \emph{Intel VT-d}. Celle d’AMD s’appelle
\emph{AMD-Vi}.
%
Une revue des options de configuration du BIOS permet de s’assure de sa
présence.
%
En cas de doute, il est possible de s’assure de son activation en amorçant le
matériel sur un système Linux quelconque et de lancer la commande :
\texttt{dmesg | grep -i dmar}.
%
Il faut s’assurer le cas échéant que le noyau Linux possède le support adéquat,
ce qui est le cas avec celui présent sur la clef USB \texttt{ChipSec}.

\section{Exigence 8}

\begin{quote}
  Si un TPM est présent, il doit être certifié (au sens des critères communs) au
  niveau EAL4+ suivant l’un des deux profils de protection suivant :
  \begin{itemize}
  \item Trusted Platform Module TPM Family 1.2 1 TCG Protection Profile PC
    Client ;
  \item Specific TPM family 2.0.
  \end{itemize}
\end{quote}

\paragraph{Motivation}
%
Le \emph{Trusted Platform Module} est un composant matériel pouvant intervenir
dans la sécurisation de la séquence d’amorçage du matériel.
%
Les TPM évalués au niveau EAL4+ ont été regardé par un laboratoire indépendant
dans le cadre d’une évaluation de sécurité de type Critères Communs.

\paragraph{Procédure}
%
Le vérificateur doit vérifier la bonne fourniture du certificat Critères Communs
par le candidat.

\section{Exigence 9}

\begin{quote}
  Le démarrage de la plate-forme doit pouvoir être protégé par un mot de passe à
  l’initiative de l’autorité qualifiée.
\end{quote}

\paragraph{Motivation}
%
L’utilisation d’un mot de passe pour protéger le démarrage de la machine reste à
ce jour l’un des moyens les plus simples de s’assurer qu’un attaquant ne peut
pas utiliser le matériel une fois éteint.

\paragraph{Procédure}
%
Le vérificateur doit vérifier que le menu de configuration du BIOS offre bien la
possibilité de protéger la séquence de démarrage du matériel par un mot de
passe.
%
Il est aussi possible d’essayer de configurer un mot de passe et de vérifier
qu’il est bien demandé au prochain redémarrage du matériel.

\section{Exigence 10}

\begin{quote}
  L’ordre de sélection des périphériques susceptibles d’amorcer le système doit
  être modifiable à l’initiative de l’autorité qualifiée.
\end{quote}

\paragraph{Motivation}
%
Configurer le matériel pour que le disque dur soit toujours sélectionné en
premier par le BIOS est une bonne pratique de sécurité.

\paragraph{Procédure}
%
Le vérificateur doit vérifier que le menu de configuration du BIOS offre bien la
possibilité de protéger la séquence de démarrage du matériel par un mot de
passe.

\section{Exigence 11}

\begin{quote}
  L’accès en modification aux paramètres du firmware (activation ou
  désactivation de fonctions, ordre de démarrage des périphériques etc.) doit
  pouvoir être protégé par un mot de passe à l’initiative de l’autorité
  qualifiée.
\end{quote}

\paragraph{Motivation}
%
L’utilisation d’un mot de passe pour protéger l’accès au menu de configuration
du BIOS est essentielle pour s’assurer qu’un attaquant n’est pas en mesure de
simplement changer cette configuration pour, par exemple, désactiver certains
mécanismes de sécurité du \emph{firmware}.

\paragraph{Procédure}
%
La vérification se fait dans le menu de configuration du BIOS.

\section{Exigence 12}

\begin{quote}
  Les mécanismes de sécurité du firmware sont à l’état de l’art au moment de la
  livraison.
  %
  En particulier, celui-ci doit respecter la norme ISO 19678 Information
  Technology - BIOS protection guidelines.
\end{quote}

\paragraph{Motivation}
%
La sécurité des \emph{firmwares} a fait l’objet d’une attention accrue ces
dernières années.
%
De nombreuses failles de sécurité ont été révélées.
%
Les processeurs les plus récents intègrent les contre-mesures adéquates, mais il
est important de vérifier que le \emph{firmware} les utilise correctement.

\paragraph{Procédure}
%
L’utilisation de l’outil Chipsec doit permettre de relever les problèmes
potentiels les plus évidents ; ces derniers demanderont une vérification plus
avant, pour s’assurer que la faiblesse révélée n’est pas contrebalancée par
d’autres contre mesures.
%
L’ANSSI fournit une image d’un système Linux minimal pouvant être copié sur une
clef USB.
%
Le système comporte l’outil Chipsec.

\section{Exigence 13 \& 14}

\begin{quote}
  Le titulaire s’engage à intégrer ce correctif dans les nouvelles machines
  livrées sous les mêmes délais.
\end{quote}

\paragraph{Motivation}
%
Le principal intérêt de ces exigences est de créer une clause de \og{}garantie
pour la sécurité\fg{}, qui permet à l’autorité qualifiée de s’assurer la
disponibilité de mises à jour de sécurité venant corriger les vulnérabilités les
plus critiques.

\paragraph{Procédure}
%
Ces exigences sont contractuelles : en postulant, le candidat s’engage à s’y
conformer.

\section{Exigence 15 \& 16}

\begin{quote}
  Les modules de firmware susceptibles de modifier le comportement du système
  d’exploitation ou l’intégrité des données (par exemple, les services résidents
  de protection contre le vol) sont soit absents du firmware, soit présents et
  désactivés par défaut.
  %
  Ces modules ne peuvent être activés que par l’autorité qualifiée (pas par
  l’utilisateur) de façon sécurisée.
  %
  La désactivation de ces modules ne doit pas être plus complexe que leur
  activation; les prérequis nécessaires à l’activation doivent être suffisants
  pour la désactivation.
  %
  Si les modules sont présents et désactivés, ils ne doivent ni initier ni
  accepter des connexions réseau.
\end{quote}

\begin{quote}
  Les dispositifs de contrôle à distance susceptibles de modifier le firmware ou
  ses données (par exemples celles utilisant \emph{AMD PSP} ou Intel
  \emph{Management Engine}, comme Intel \emph{Active Management Technology}
  (AMT), qui échappent au contrôle du système d’exploitation, sont soit absents
  du firmware, soit présents et désactivés par défaut.
  %
  Ces modules ne peuvent être activés que par l’autorité qualifiée (pas par
  l’utilisateur) de façon sécurisée.
  %
  La désactivation de ces modules ne doit pas être plus complexe que leur
  activation; les prérequis nécessaires à l’activation doivent être suffisants
  pour la désactivation.
  %
  Si les modules sont présents et désactivés, ils ne doivent ni initier ni
  accepter des connexions réseau.
\end{quote}

\paragraph{Motivation}
%
Sans doute dans un but de différentiation par rapport à leurs concurrents, les
constructeurs ont eu tendance ces dernières années à développer des solutions de
management à distance.
%
Ces dernières posent un important risque pour la sécurité de la plate-forme.
%
Il est donc préférable qu’elles soient \textbf{(1)} connues de l’autorité
qualifiées et \textbf{(2)} désactivées par défaut.

\paragraph{Procédure}
%
Le vérificateur peut d’abord aller regarder dans les menus de configuration du
BIOS si de tels modules (type typiquement, les modules du \emph{Management
  Engine} d’Intel) y apparaissent et sont/peuvent être désactivés.
%
La plupart de ces services ayant accès à l’interface réseau, il serait sans
doute intéressant de brancher le matériel testé sur un réseau contrôlé et de
capturer son trafic sortant.

\section{Exigence 17}

\begin{quote}
  Les mécanismes de protection contre l’inspection de code sont soit absents du
  firmware, soit présents et désactivés par défaut.
  %
  Ces modules ne peuvent être activés que par l’autorité qualifiée (pas par
  l’utilisateur) de façon sécurisée.
  %
  La désactivation de ces modules ne doit pas être plus complexe que leur
  activation; les prérequis nécessaires à l’activation doivent être suffisants
  pour la désactivation.
\end{quote}

\paragraph{Motivation}
%
Dans une logique de maîtrise du matériel qu’elle utilise, l’autorité devrait
avoir la possibilité d’inspecter le \emph{firmware} installé sur ce dernier.

\paragraph{Procédure}
%
Un relevé de configuration de la Flash SPI (par exemple grâce à Chipsec) permet
de savoir si le contenu est bien protégé en écriture, mais aussi de savoir s’il
est accessible en lecture.

\section{Exigence 18}

\begin{quote}
  L’outil de configuration du firmware UEFI doit proposer un moyen de
  remplacement des clés SecureBoot du titulaire et des tierces-parties par des
  clés fournies par l’autorité qualifiée (pas par l’utilisateur).
\end{quote}

\paragraph{Motivation}
%
Le \emph{Secure Boot} est un mécanisme de sécurité logicielle très appréciable
pour la sécurisation de la séquence de démarrage d’un matériel.
%
Cependant, il peut être détourné de son utilité première à des fins de
verrouillage de la machine.
%
Cette exigence sert à prévenir ce scénario.

\paragraph{Procédure}
%
La vérification se fait manuellement en allant vérifier dans les menus de
configuration du BIOS.
%
L’ANSSI fournit le matériel cryptographique et les utilitaires nécessaires pour
tester le remplacement des clefs SecureBoot d’une machine, sous la forme d’une
image pouvant être copiée sur une clef USB.

\section{Exigence 19}

\begin{quote}
  Si un TPM est présent dans la configuration, il ne peut être activé que par
  l’autorité qualifiée (pas par l’utilisateur) de façon sécurisée.
  %
  La désactivation de celui-ci ne doit pas être plus complexe que son
  activation; les prérequis nécessaires à l’activation doivent être suffisants
  pour la désactivation.
\end{quote}

\paragraph{Motivation}
%
L’autorité qualifié doit avoir la maîtrise du matériel qu’elle se procure.

\paragraph{Procédure}
%
La vérification se fait manuellement en allant vérifier dans les menus de
configuration du BIOS.

\section{Exigence 20}

\begin{quote}
  La plate-forme dispose d’une protection contre la modification malveillante du
  firmware (par exemple, un verrou d’accès en lecture seule) ou d’un mécanisme
  de détection des modifications (par exemple, une vérification de signature),
  accompagnés d’une fonction de notification fiable à l’autorité qualifiée ou à
  l’utilisateur de la plate-forme.
  %
  On inclut dans les modifications malveillante le fait d’installer une version
  inférieure du firmware, par exemple pour réinstaller une version correctement
  signée mais vulnérable.
\end{quote}

\paragraph{Motivation}
%
Il est important de ne pas laisser à l’attaquant la possibilité de forcer
l’installation d’un \emph{firmware} arbitraire ou plus ancien que la version
courante.

\paragraph{Procédure}
%
La vérification se fait manuellement en allant vérifier dans les menus de
configuration du BIOS.
%
En plus de la bonne configuration de la flash SPI (\textsc{ChipSec}), la
vérification devrait aussi comprendre la tentative d’installation d’une ancienne
version du BIOS de la machine sélectionnée (la plupart des constructeurs ont un
site Internet public depuis lequel il est possible de télécharger différentes
version du firmware d’une machine).
%
La tentative d’installation d’une version de firmware dont la signature a été
altérée est sans doute nécessaire dans un souci d’exhaustivité.

\section{Exigence 21}

\begin{quote}
  Les composants présents sur la plate-forme (interfaces de communications,
  ports d’entrée sortie, périphériques internes) doivent pouvoir être désactivés
  dans l’interface de configuration du firmware par l’autorité qualifiée (pas
  par l’utilisateur) de façon sécurisée.
  %
  La désactivation de ces modules ne doit pas être plus complexe que leur
  activation; Les prérequis nécessaires à l’activation doivent être suffisants
  pour la désactivation.
  %
  Si les modules sont désactivés, ils ne doivent ni initier ni accepter des
  communications ou connexions réseau.
\end{quote}

\paragraph{Motivation}
%
Cette exigence s’inscrit dans la même logique que les exigences précédentes
demandant la possibilité de désactiver depuis l’interface du BIOS des composants
matériels: limiter la surface d’attaque et s’assurer la maîtrise du matériel
obtenu.

\paragraph{Procédure}
%
La vérification se fait manuellement en allant vérifier dans les menus de
configuration du BIOS.


\chapter{Exigences logicielles}
% -----------------------------------------------------------------------------

\section{Exigence 22}

\begin{quote}
  Les plateformes n’imposent pas le support d’un système d’exploitation
  particulier, ni d’une version précise de celui-ci.
\end{quote}

\paragraph{Motivation}
%
Il est dans l’intérêt de l’autorité qualifiée de conserver \emph{in fine} le
choix final du système d’exploitation qu’elle utilise sur les matériels qu’elle
se procure.

\paragraph{Procédure}
%
La vérification de certaines exigences du CCTP demandant de démarrer sur une
clef USB possédant un système GNU/Linux, son bon démarrage est une vérification
suffisante dans le cadre des tests de conformité.


% =============================================================================

\appendix

% =============================================================================
\part{Annexes}

\chapter{Modèle de liste des composants} \label{annexe:modele}
% -----------------------------------------------------------------------------

\chapter{Formulaire}
% -----------------------------------------------------------------------------

\begin{Form}
  \q{Vérificateur}
  \q{Candidat}
  \q{Référence Matériel}
\end{Form}

\section{Exigences \og{}documentaires\fg{}}

Pour satisfaire aux exigences documentaires, le candidat doit fournir, sous
forme de listes, les composants matériels de sa plate-forme; pour l’aider dans
sa démarche, il est conseillé de lui fournir le document décrit dans
l’Annexe~\ref{annexe:modele}.

En plus de cette liste, il convient de vérifier que sont bien fournis:

\begin{Form}
  \begin{tabularx}{\textwidth}{Xcc}
    & Oui & Non \\
    Le certificat Critères Communs du TPM
       & \CheckBox[width=1em]{}
       & \CheckBox[width=1em]{} \\
    Les rapports d’évaluation suivant les nomes EN 55024 et EN 55022
       & \CheckBox[width=1em]{}
       & \CheckBox[width=1em]{}
  \end{tabularx}
\end{Form}

\section{Exigences \og{}configuration\fg{}}

La vérification de ces exigences est rapide dans la plupart des cas, mais peut
être compliquée si le menu de configuration du BIOS n’est pas accessible
facilement\,\footnote{Par exemple, certains ordinateurs HP ne font pas la
  publicité de la séquence de touches nécessaires pour entrer dans le menu de
  configuration du BIOS.} ou si sa présentation n’est pas habituelle.
%
Dans ces cas là, il ne faut pas hésiter à solliciter le candidat directement.

\begin{Form}
  \begin{tabularx}{\textwidth}{Xcc}
    & Oui & Non \\
    Présence d’une IO/MMU (VT-d, AMD-Vi) activée par défaut
      & \CheckBox[width=1em]{}
      & \CheckBox[width=1em]{} \\
    Choix d’un mot de passe de démarrage
      & \CheckBox[width=1em]{}
      & \CheckBox[width=1em]{} \\
    Choix d’un mot de passe pour le menu de configuration BIOS
      & \CheckBox[width=1em]{}
      & \CheckBox[width=1em]{} \\
    Choix de l’ordre de sélection des périphériques au démarrage
      & \CheckBox[width=1em]{}
      & \CheckBox[width=1em]{} \\
    Modules de gestion à distance (\emph{AMT}, \emph{Computrace},
    \emph{Anti-Theft}) désactivés par défaut
      & \CheckBox[width=1em]{}
      & \CheckBox[width=1em]{} \\
    Possibilité de désactiver les interfaces de communication
      & \CheckBox[width=1em]{}
      & \CheckBox[width=1em]{} \\
    Possibilité de désactiver les ports d’entrée/sortie (par exemple USB)
      & \CheckBox[width=1em]{}
      & \CheckBox[width=1em]{} \\
  \end{tabularx}
\end{Form}

\section{Exigences \og{}interfaces\fg{}}

Dans une visée de réduction de la surface d’attaque potentielle, les matériels
informatiques livré dans le cadre du marché doivent satisfaire aux exigences
suivantes :

\begin{Form}
  \begin{tabularx}{\textwidth}{Xcc}
    & Oui
    & Non \\
    L’interface WiFi est démontable sans perturber le bon fonctionnement de la
    machine
    & \CheckBox[width=1em]{}
    & \CheckBox[width=1em]{} \\
    Les interfaces qui ne sont pas listées dans la description de la
    configuration matérielle (\emph{Bluetooth}, 3G et 4G, etc.) sont désactivées
    par défaut
    & \CheckBox[width=1em]{}
    & \CheckBox[width=1em]{} \\
  \end{tabularx}
\end{Form}


\section{Exigences \og{}durcissement\fg{}}

La vérification des exigences concernant le durcissement du BIOS permet de
vérifier que le candidat ne fait pas d’erreurs trop importantes.
%
Malheureusement, la diversité des solutions propriétaires utilisées par les
constructeurs rend difficile la proposition d’une procédure précise poussée.
%
Pour cette raison, l’ANSSI a pris le parti d’utiliser
\textsc{ChipSec}\,\footnote{\url{https://github.com/chipsec/chipsec}}, un outil
\emph{open source} proposé par Intel.

\begin{enumerate}
\item Amorcer le matériel sur la clef USB \texttt{ChipSec} fournit par l’ANSSI
\item Se connecter avec l’utilisateur \texttt{root}
\item Lancer \textsc{ChipSec} en tapant dans l’invite de commande
  \texttt{chipsec\_main}
\end{enumerate}

\begin{Form}
  \begin{tabularx}{\textwidth}{Xcc}
    & Oui & Non \\
    La commande \texttt{chipsec\_main}, une fois terminée, n’annonce aucun échec
      & \CheckBox[width=1em]{}
      & \CheckBox[width=1em]{} \\
  \end{tabularx}
  \q{Si \og{}Non\fg{} lesquels :}
  \begin{tabularx}{\textwidth}{Xcc}
    La commande \texttt{chipsec\_util spi dump img.bin} termine sans erreur
      & \CheckBox[width=1em]{}
      & \CheckBox[width=1em]{} \\
  \end{tabularx}
\end{Form}

L’utilisation de {\sc ChipSec} n’est possible qu’avec certains matériels
Intel.
%
Il est donc possible que l’outil ne fonctionne pas correctement avec certains
processeurs Intel peu communs ou avec du matériel basé sur AMD.
%
En premier recours, il est possible de forcer le lancement de {\sc ChipSec} avec
la command \texttt{chipsec\_main -i}.
%
En second recours, l’ANSSI se tient à la disposition des vérificateurs.

\section{Exigences \og{}\emph{Secure Boot}\fg{}}

Le constructeur doit laisser la possibilité à l’autorité qualifiée de changer
les clefs \emph{Secure Boot} utilisées par le BIOS pour vérifier les signatures
des composants logiciels utilisés lors de la séquence de démarrage du matériel.
%
Sans cette possibilité, il n’est potentiellement pas possible d’avoir la
maîtrise du système d’exploitation s’exécutant sur le matériel.
%
Le \emph{Setup Mode} est un état particulier du BIOS qui permet de changer les
clefs utilisées par le \emph{Secure Boot}.
%
Certains BIOS ne fournissent pas ce mode, et préfèrent proposer des interfaces
de configuration directement dans la configuration du BIOS.
%
Le cas échéant, il ne faut pas hésiter à demander des précisions au candidat.

\begin{enumerate}
\item Entrer dans le menu de configuration du BIOS
  \begin{enumerate}
  \item Trouver la section relative au \emph{Secure Boot} (souvent dans un
    onglet \emph{Security})
  \item Désactiver le \emph{Secure Boot}
  \item Activer le \emph{Setup Mode} (par exemple, Lenovo possède une option
    \emph{Reset to Setup Mode})
  \end{enumerate}
\item Redémarrer et amorcer la séquence de démarrage de matériel sur la clef USB
  \texttt{SecBoot} fournie par l’ANSSI
  \begin{enumerate}
  \item Lorsque l’invite de commande du \emph{Shell EFI} apparaît, taper
    \texttt{fs0:}, puis \texttt{dir}
  \item Si l’utilitaire \texttt{KeyTool.efi} apparaît dans la liste, continuer,
    sinon revenir à l’étape précédente en incrémentant l’entier après
    \texttt{fs}
  \item Lancer \texttt{KeyTool.efi}
  \item Sélectionner \emph{Edit Keys}
  \item Sélectionner \emph{Allowed Signature Database (db)}
  \item Sélectionner le premier disque proposé, chercher \texttt{db.esl}; s'il
    est présent dans la liste, le sélectionner, sinon appuyer sur
    \texttt{Echap.} et sélectionner le disque suivant, jusqu’à le trouver
  \item Sélectionner \emph{The Key Exchange Key (KEK)}
  \item Sélectionner le bon disque, puis \texttt{KEK.esl}
  \item Sélectionner \emph{The Platform Key (PK)}
  \item Sélectionner le bon disque, puis \texttt{PK.auth}
  \item Quitter l’application \texttt{KeyTool.efi}
  \end{enumerate}
\end{enumerate}

Pour vérifier le bon déroulé des étapes précédentes, il est possible d’essayer
d’exécuter des applications UEFI depuis le \emph{Shell EFI}.

\begin{Form}
  \begin{tabularx}{\textwidth}{Xcc}
    & Oui & Non \\
    L’application \texttt{KeyTool.efi} refuse de démarrer
      & \CheckBox[width=1em]{}
      & \CheckBox[width=1em]{} \\
    L’application \texttt{HelloWorld.efi} fonctionne correctement
      & \CheckBox[width=1em]{}
      & \CheckBox[width=1em]{} \\
  \end{tabularx}
\end{Form}

\textbf{Attention :} Si le vérificateur essaie de lancer l’outil
\textsc{ChipSec} après avoir configuré \emph{Secure Boot}, il est possible que
l’un de ses tests échoue.

% =============================================================================

\end{document}